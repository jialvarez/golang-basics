\section{Flow control}
\begin{frame}[fragile]
\frametitle{For}

\begin{lstlisting}
package main

import "fmt"

func main() {
    sum := 0
    for i := 0; i < 10; i++ { // init and post are optional
        sum += i
    }
    fmt.Println(sum)

    newsum := 1 // Go doesn't like to reuse vars
    for newsum < 1000 {
        newsum += newsum
    }
    fmt.Println(newsum)
}
\end{lstlisting}

\end{frame}


\begin{frame}[fragile]
\frametitle{If}
\begin{lstlisting}
package main

import (
    "fmt"
    "math"
)

func pow(x, n, lim float64) float64 {
    if v := math.Pow(x, n); v < lim {
        return v
    } else {
        fmt.Printf("%g >= %g\n", v, lim)
    }
    // can't use v here, though
    return lim
}

func main() {
    fmt.Println(
        pow(3, 2, 10), pow(3, 3, 20),
    )
}
\end{lstlisting}
\end{frame}

\begin{frame}[fragile]
\frametitle{Switch}
\begin{lstlisting}
package main

import (
    "fmt"
    "runtime"
)

func main() {
    fmt.Print("Go runs on ")
    switch os := runtime.GOOS; os {
    case "darwin":
        fmt.Println("OS X.")
    case "linux":
        fmt.Println("Linux.")
    default:
        // freebsd, openbsd,
        // plan9, windows...
        fmt.Printf("%s.\n", os)
    }
}
\end{lstlisting}
\end{frame}

\begin{frame}[fragile]
\frametitle{Defer}
\begin{lstlisting}
package main

import "fmt"

func main() {
    fmt.Println("counting")

    for i := 0; i < 10; i++ {
        defer fmt.Printf("%v,",i)
    }

    fmt.Println("done")
}

counting
done
9,8,7,6,5,4,3,2,1,0,
\end{lstlisting}
\end{frame}