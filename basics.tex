\section{Basics}
\begin{frame}[fragile]
\frametitle{Basics}

\begin{lstlisting}
package main

import "fmt"

var a = 2

func swap(x, y string) (string, string) {
    return y, x
}

func main() {
    a, b := swap("hello", "world")
    fmt.Println(a, b)

    var i, j int = 1, 2
    k := 3
    c, python, java := true, false, "no!"

    fmt.Println(i, j, k, c, python, java, a)
}
\end{lstlisting}
\end{frame}

\begin{frame}
\frametitle{Basics}

	\begin{itemize}
		\item Packages
		\item Imports
		\item Exported names
		\item Functions
		\item Multiple return results
		\item Variables at package or function level and with initializers
		\item Short variable declarations
	\end{itemize}

\end{frame}


\begin{frame}
\frametitle{Basics}


\begin{itemize}
	\item Go's basic types are:

	\begin{itemize}
		\item bool
		\item string
		\item int  int8  int16  int32  int64 uint uint8 uint16 uint32 uint64 uintptr
		\item byte // alias for uint8
		\item rune // alias for int32, represents a Unicode code point
		\item float32 float64
		\item complex64 complex128
	\end{itemize}

	\item And for zero values:

	\begin{itemize}
		\item 0 for numeric types,
		\item false for the boolean type, and
		\item "" (the empty string) for strings.
	\end{itemize}
	
\end{itemize}
\end{frame}

\begin{frame}[fragile]
\frametitle{Basics}

\begin{lstlisting}
package main

import (
    "fmt"
    "math"
)

const Pi = 3.14

func main() {
    var x, y int = 3, 4
    f := math.Sqrt(float64(x*x + y*y))
    z := uint(f)
    fmt.Println(x, y, z)

    var i int
    j := i // j is an int

    i := 42           // int
    f := 3.142        // float64
    g := 0.867 + 0.5i // complex128
}

\end{lstlisting}
\end{frame}

\begin{frame}
\frametitle{Basics}

\begin{itemize}
	\item Type conversions
	\item Type inference
	\item Constants
	\item Numeric constants
\end{itemize}

\end{frame}
